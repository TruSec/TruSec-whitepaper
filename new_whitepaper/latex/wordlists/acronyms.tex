% Nomenclature is a list of abbreviations and their description.
% Source: https://www.overleaf.com/learn/latex/Glossaries
% All before begin document
%\usepackage[utf8]{inputenc}
%\usepackage[acronym,toc]{glossaries}

\newacronym{alu}{ALU}{Arithmatic Logic Unit}
\newacronym{ann}{ANN}{Artificial Neural Network}
\newacronym{bit}{BIT}{Binary Digit}
\newacronym{biser}{BISER}{Built-in Soft Error Resillience}
\newacronym{bios}{BIOS}{Basic Input/Output System}
\newacronym{bjt}{BJT}{Bipolar Junction Transistor}
\newacronym{cme}{CME}{Coronal Mass Ejection}
\newacronym{cmos}{CMOS}{Complementary Metal–Oxide–Semiconductor}
\newacronym{cpu}{CPU}{Central Processing Unit}
\newacronym{dec}{DEC}{Double-error Correcting Code}
\newacronym{dot}{DOT}{Design Options Structuring Tree}
\newacronym{esa}{ESA}{European Space Agency}
\newacronym{dice}{DICE}{Dual Interlocked Storage Cell}
\newacronym{dnn}{DNN}{Deep Neural Network}
\newacronym{dnu}{DNU}{Dual-node Upset}
\newacronym{dod}{DOD}{Department of Defence}
\newacronym{dram}{DRAM}{Dynamic Random Access Memory}
\newacronym{ecc}{ECC}{Error Correction Code}
\newacronym{elt}{ELT}{Enclosed Layout Transistor}
%\newacronym{enc}{ENC}{Effective Nuclear Charge}
%https://en.wikipedia.org/wiki/Effective_nuclear_charge
\newacronym{fet}{FET}{Field-Effect Transistor}
\newacronym{ffd}{FFD}{Functional Flow Diagram}
\newacronym{fbd}{FBD}{Functional Break-down Diagram}
\newacronym{gcd}{GCD}{Greatest Common Divisor}
\newacronym{gcr}{GCR}{Galactic Cosmic Ray}
\newacronym{ic}{IC}{Integrated Circuit}
\newacronym{inrc}{INRC}{Intel Neuromorphic Research Community}
\newacronym{lcm}{LCM}{Least Common Multiple}
\newacronym{let}{LET}{Linear Energy Transfer}
\newacronym{mpnn}{MPNN}{Multilayer Perceptron Neural Network}
\newacronym{mos}{MOS}{Metal–Oxide–Semiconductor}
\newacronym{mosfet}{MOSFET}{Metal–Oxide–Semiconductor Field-Effect Transistor}
\newacronym{msn}{MSN}{Mission Need Statement}
\newacronym{nvm}{NVM}{Non-Volatile Memory}
\newacronym{pos}{POS}{Project Objective Statement}
\newacronym{ram}{RAM}{Random Access Memory}
\newacronym{rdt}{RDT}{Requirements Discovery Tree}
\newacronym{rom}{ROM}{Read Only Memory}
\newacronym{sec}{SEC}{Single-error Correcting Code}
\newacronym{see}{SEE}{Single-event Effects}
\newacronym{seib}{SEIB}{Single-event Induced Burnout}
\newacronym{segr}{SEGR}{Single-event Gate Rupture}
\newacronym{sel}{SEL}{Single-event Latch-up}
\newacronym{ser}{SER}{Soft-error Rate}
\newacronym{serl}{SERL}{Soft-error Resilient Latch}
\newacronym{set}{SET}{Single-event Transient}
\newacronym{seu}{SEU}{Single-event Upset}
\newacronym{ses}{SES}{Single-event Snapback}
\newacronym{snn}{SNN}{Spiking Neural Network}
\newacronym{snu}{SNU}{Single-node Upset}
\newacronym{sram}{SRAM}{Static Random Access Memory}
\newacronym{spe}{SPE}{Solar Particle Events}
\newacronym{heynderickx_new_2004}{SPENVIS}{Space Environment Information System}
\newacronym{stdp}{STDP}{Spike-Timing-Dependent Plasticity}
\newacronym{tlb}{TLB}{Translation Lookaside Buffer}% TODO: add to glossary
\newacronym{tmr}{TMR}{Tripple-mode Redundancy}
\newacronym{vlsi}{VLSI}{Very-Large-Scale Integration}
\newacronym{wfd}{WFD}{Work Flow Diagram}
\newacronym{wbs}{WBS}{Work Break-down Structure}
%\newacronym{}{}{}
%\printglossary