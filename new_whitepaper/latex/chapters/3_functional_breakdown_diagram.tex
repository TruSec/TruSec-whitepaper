\chapter{Functional Breakdown Diagram}\label{chap:baseline_fbd}
This section presents the functional breakdown diagram of the system that is designed in this thesis project. This \acrshort{fbd} is generated using the detailed functional flow diagram of \cref{sec:baseline_detailed_ffd}. The \acrshort{fbd} presents the activities in an hierarchical style.

\begin{enumerate}
    \item Run space-related function on neuromorphic architecture.
    \begin{enumerate}
        \item Initialise neuromorphic architecture.
        \item Optional: Load brain adaptation implementation.
        \item Load space related function.
        \item Load space related function data.
        \item Run space related function.
        \item Complete running space related function.
        \item Retrieve space related function outputs.
        \item Convert space related function outputs to performance score.
        \item Report difference between the scores of the architectures with- and without brain adaptation.
        \item Determine significance of difference.
    \end{enumerate}
    \item Render and endure modelled space radiation on neuromorphic architecture.
    \begin{enumerate}
        \item Generate simulated space radiation pattern of neuromorphic space architecture.
        \item Expose neuromorphic architecture to simulated space radiation pattern.
        \item Optional: model architecture-radiation interaction.
        \item Optional: measure architecture-radiation interaction.
    \end{enumerate}
\end{enumerate}