\section{Assumptions}\label{sec:assumptions}
This sections present some of the assumptions that are made about the current way zero-day exploits, and vulnerabilities in general are treated.

\subsection{Ethical Hacker Perspective}
\begin{enumerate}
	\item We assume it is not always as easy and/or attractive for whitehat hackers/ethical hackers to publish an exploit and retrieve an accompanying financial reward for the publication. This assumption is based on popular media such as darknet diaries, posts on \url{news.ycombinator.com}, communications with two ethical hackers and possibly other sources. This assumption is based on (a combination of) the following sub-assumptions:
	\begin{enumerate}
		\item Vulnerabilities may be discovered at small/non-profit software development companies that have not allocated a large budget fraction to security.
		\item Ambiguity in the specification of the bug bounty/reward program may be interpreted in the advantage of the company during triage.
		\item The triage process may take a relatively long time, requiring the ethical hacker to have sufficient funds to sustain living costs coverage until the pay-out.
		\item A conservative/carefulness in the ethical hacker towards approaching the company with respect to the legality of discovering the vulnerability may hinder/slow down the vulnerability disclosure process.
		\item The effort required to contact the company and convince them of the seriousness of the bug may consume unnecessary resources.
	\end{enumerate}
\end{enumerate}

\subsection{Company - perspective}
\begin{enumerate}
	\item We assume cybersecurity vulnerabilities become increasingly more relevant in our increasingly more digitized world. This assumption may be seen as being substantiated by for example the \textit{Cyber Security Assessment Netherlands 2021 (CSAN 2021)} as presented by the Dutch National Coordinator Counterterrorism and Safety of the Ministry of Justice and Security. Currently, there is only the Dutch version available at: \url{https://www.nctv.nl/onderwerpen/cybersecuritybeeld-nederland/documenten/publicaties/2021/06/28/cybersecuritybeeld-nederland-2021}. We assume that this trend can be extrapolated from a Dutch perspective to a more global perspective, given the international media coverage of many ransomware attacks.
	\item We assume that companies are interested, or will become more interested, in showing their customers and/or stakeholders (a quantified perspective on) \textit{how} secure their technology is. We assume it can be quite challenging to convey this perspective clearly due to the following factors:
	\begin{enumerate}
		\item Vulnerabilities can be found in various sections of the company, ranging from social engineering, misconfiguration to zero-day exploits. It is difficult to give customers a comprehensive yet concise/simple insight in how "secure" all these attack surfaces are.
		\item The impact of a vulnerability may be ambiguous or not easily quantifiable. For example, for some companies, vulnerabilities may allow malicious actors to take over critical infrastructure, whilst other vulnerabilities may lead to data leaks or other undesired side effects.
		\item It may be difficult to accurately assess the capabilities of malicious adversaries.
	\end{enumerate}
	\item We assume some companies might be unfamiliar with vulnerability disclosure and accompanying triage processes. These delicate processes may seem intimidating for new companies that want to start paying attention to their cybersecurity, and this may lead to a lower allocation of cybersecurity budget.
\end{enumerate}