\section{Conclusion and Recommendations}
\label{sec:conclusion}
The proposed protocol can enable companies to convey a quantitative level of security of (segments of) their open source technology stack to their customers. Customers can use this information on the minimum price for zero-day exploits, and compare it to their costs of suffering from a malicious zero-day exploit. This may allow them to (re)allocate their funds and cyber-security resources based on the accompanying risk-profile. 

Additionally, the protocol enables ethical hackers to retrieve payouts directly without ambiguity.

Since computational budgets typically are costly on decentralised computing platforms, a recommendation is included to investigate the option to allow staking decentralised containerized applications instead of complete decentralised virtual machines.
%% Summarize the other chapters
%% Explain what TruSec does

%% Write how TruSec works
%% Summarise implementation
%% Summarize security issues
%% Summarise discussion
%% Mention the protocol can promote a fair and inclusive TDD market contributing to SDG 8.



%\subsection{Recommendations}
%% Include the possible impact of the protocol on society
%The following recommendations are made:
%\begin{enumerate}[label=\arabic*]
    %% To mitigate edge cases
%    \item \label{rec:ecosystem} 
%    \item \label{rec:prelauch_period}
%    \item \label{rec:security_tools}
%\end{enumerate}