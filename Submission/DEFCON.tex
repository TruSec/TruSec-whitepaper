\documentclass{article}
\usepackage[utf8]{inputenc}

\title{2022 DEF CON USA submission}
\author{TruCol}

\begin{document}
\maketitle
\section{Introduction}
This document contains the requirements for the 2022 DEF CON conference talk submission for the TruSec protocol, as well as the data that is entered for the submission to give the talk at DEF CON.
\section{Requirements}
% https://media.defcon.org/DEF%20CON%2029/DEF%20CON%2029%20workshops/DEF%20CON%2029%20Workshop%20Philippe%20Delteil%20Bug%20Bounty%20Workshop.pdf
% https://defcon.org/html/defcon-30/dc-30-cfp.html
\begin{enumerate}
    \item CFP forms and questions should get mailed to: talks/at/defcon.org.
    \item If you have not received confirmation of your submission after two business days, contact us again. Please check your spam folder & whitelist */at/defcon.org. 
    \item After a completed CFP form is received, speakers will be contacted if there are any questions about their presentations.
    \item Source: Call For Papers form (CFP form): \url{https://defcon.org/html/defcon-30/dc-30-cfp-form.html}
    \item If you wish to send your submission encrypted via PGP please use KeyID:3A276723 for Nikita(at)DEFCON(dot)org, Found here: NIKITA PGP

\end{enumerate}


\section{Submission}
\subsection{Presenter Information}
Example Presentation style:
% https://www.youtube.com/watch?v=799XIwqf_hU
\subsection{Presentation Information}

===== BEGIN CFP FORM =====
%PRESENTER INFORMATION
%Primary Speaker Name or Pseudonym*:
Victoria Bosch
%Primary Speaker's preferred pronouns:
She
%Primary Speaker Title and Company (if applicable):
TruCol
%Additional Speaker Name(s)/Pseudonym(s)* and Title(s) and preferred pronouns:
None.
%Have any of the speakers spoken at a previous DEF CON? If Yes, which speaker, what year(s), and which talk(s)?
No.
%Primary Speaker Email Address:
TODO
%Backup Email, in case of communication failure (optional):
a.h.h.toeter@student.tudelft.nl
%Primary Speaker Phone Number:
TODO
%Co-Speaker(s) Email Address(es):
a.h.h.toeter@student.tudelft.nl
%Co-Speaker(s) Phone Number(s):
0031 6 4182 4424
%Speakers' Social Media / Personal Site Information (Twitter, Facebook, website, etc)*:
https://trucol.io/
https://nl.linkedin.com/in/victoria-bosch
https://www.linkedin.com/in/akke-t

%Would you like your submission sent in anonymously to the review board?*: Yes or No
No

%NEW! What do you bring to the table, how will you or your presentation(s) contribute a new perspective to the content at DEF CON? (Please answer for each speaker &/or panel members.)
We intend to present a protocol that can solve 3 problems:
0. Ambiguity is removed in bounty payouts to ethical hackers for deterministically verifiable zero-day exploits.
1. Companies gain quantitative insight in the level of security of staked open source stacks against zero-day exploits.
2. Manual labour of cybersecurity triage can be eliminated for deterministically verifiable zero-day exploits.

%NEW! If your presentation is accepted, do you consent to inviting the primary & secondary contacts to the official DEF CON Speaker's planning project on "Basecamp"? We will use the email addresses you included on your application (unless stated otherwise). DEF CON staff, and other speakers would be able to see your email address once you accepted the invitation. Yes/No
Yes
%PRESENTATION INFORMATION
%Date of Submission:
2022-03-19

%Presentation Title:
Eliminating triage intermediaries for zero-day exploits using a decentralised payout protocol.

%Length of presentation: (20 minutes, 45 minutes, 105 minutes)
%Please note: We accept few 105 minute talks, your submission will need to breakdown the time requested within your content outline.
20 minutes

%Is there a demonstration? Yes or No. If yes, please explain the demonstration.
No
%Is there audience participation*? Yes or No. If yes, how?
No
%Are you releasing a new tool? Yes or No
No
%If you are releasing a new tool:
%Under which license?
The presented protocol is proposed to be implemented using fully open source code.
%Is it a full tool suite, a plugin, proof of concept exploit, update to existing tool, or something else?
We present a protocol without implementation.
%Is the tool currently public?
A whitepaper will be publicised.
%What language is the tool written in?
Not applicable.
%Please introduce your tool briefly and explain what it does:
The protocol allows companies to stake decentralised virtual machines with open source software stacks, whilst allowing hackers to deploy an attack to those decentralised VM's and automatically retrieve the bounty.

%Are you releasing a new exploit? Yes or No
No
%If you are releasing vulnerabilities, please break them down and abstract as a vulnerability database would. Include the vendor(s), product(s), and version(s) affected here:
Not applicable.

%Are there any contingencies that might prevent releasing the exploit (e.g. vendor relationships, coordinated disclosure delays, visa approvals, etc)? Yes or No
No.

%Have you submitted or do you plan on submitting this topic to any other conference held prior to DEF CON? If yes, please list which conference(s) and their dates.
Blackhat USA 2022. August 6-11.

%Are you submitting this or any other topic to Black Hat USA? Yes or No. If Yes, please elaborate.
Yes, for hackers to maximally benefit from this protocol, we need adoption of large companies, hence, we also aim to present this protocol to enterprise at blackhat.

%If you answered yes to the previous question, can you ONLY commit to speak at DEF CON if your talk is accepted at Black Hat USA? Yes or No
No
%Note: For clarification, this means if your submission to Black Hat's CFP is rejected, you plan to withdraw your DEF CON Submission.

%Are you submitting this or any other topic to BSidesLV? Yes or No. If Yes, please elaborate.
No

%Sometimes a good CFP makes a great fit for a DEF CON Village too. The review board will recommend some talks to be presented exclusively in a village, or as an encore to a main stage presentation. Do you consent to allow DEF CON to forward your completed submission to relevant DEF CON Village lead(s) for their consideration for village content? * Yes or No.
Yes

%ABSTRACT
%(not to exceed 1337 characters):
We present a protocol that collectivises security bounties for deterministically verifiable zero-day exploits(DVZEs). It helps hackers retrieve bounties without ambiguity and removes subjectivity and manual labour of triage processes for those bounties. Companies can show customers how secure their open source software is, in terms of dollars staked on the software.

Companies and users can pool bounties on open-source security stacks in decentralised virtual machines that contain a read and/or write secret. They can also specify a minimum responsible disclosure duration as well as a public key. Next, hackers can submit an attack to such a decentralised VM (DVM), by storing it in a decentralised encrypted locker (DEL), and notify the DVM of its presence. Once the stakeholders see this notification, (along with the rest of the world), they can use their private key to retrieve the attack from the DEL (before the rest of the world). For each bounty placed on the DVM, a call is made to the DEL just before the end of the accompanying responsible disclosure time. This call verifies that the attack is still encrypted. After the respective responsible disclosure periods have passed, the DEL is decrypted and the attack is executed. Successful attacks compromise the DVM read/write secret, triggering bounty hunter payout.

%Note: Your abstract will be used for the website and printed materials. Summarize what your presentation will cover. Attendees will read this to get an idea of what they should know before your presentation, and what they will learn after. Use this to inform attendees about how technical your talk is, what tools will be used, what materials to read in advance to get the most out of your presentation. This abstract is the primary way people will be drawn to your session, but should not give them the entire content of the talk. We request you keep your abstract well over 140 characters, but at or under the 1337 maximum.

%SPEAKER BIO(S)
%(not to exceed 1337 characters total):
A motivated student with an interdisciplinary bachelor in artificial intelligence and philosophy at the Liberal Arts & Sciences faculty of Utrecht University. Victoria currently works on a master in Artificial Intelligence with a specialisation in cognitive computing at Radboud University. She has a passion for making science accessible and works on science outreach as the editor in chief at De Focus” - an online platform for scientific journalism. Besides her studies, she worked on, and presented the TruCol protocol during the Ethereum conference in Paris last summer. Additionally, she co-authored the 2021 publication titled: "Implementation of a distributed minimum dominating set approximation algorithm in a spiking neural network".

%Note: This text will be used for the website and printed materials and should be written in the third person. Cover any professional or hacker history that is relevant to you and the presentation, you may include past jobs, tools that you have written, etc. Let people know who you are and why you are qualified to speak on your topic. If you prefer, you may write your bio under a pseudonym. Please include Co-Speaker and Panel Bios when possible.

%REFERENCES:
%Please provide a simple bibliography and/or works cited. List sources you have used (whether referenced or not) in the process of finalizing your presentation. Please remember to credit prior works and acknowledge others. References will be posted online with your talk information. We want attendees interested in your talk to be able to research what has been helpful for you in developing this presentation.
The protocol was derived from TruCol which can be inspected at https://github.com/TruCol
%
%DETAILED OUTLINE:
%Note: This is the most important section on the application. You must provide a detailed outline containing the main points and navigation through your talk - show how you intend to begin, where you intend to lead the audience and how you plan to get there. Your outline should be in simple text. Please do not submit slides, Docs, or PDFs as an outline. If you are submitting a panel, it’s encouraged to list what each panel member will contribute to on the outline, as well as the estimated time budgeted for those bullet points. The review board likes submissions that include references to prior works and research you used in developing your presentation. The more detailed your outline then the better we are able to review your presentation against other submissions (and the higher chance you have of being accepted).
I. State assumptions: 
A. Mention why we assume the world could benefit from seeing how much money is staked on the security of all open source software stacks. 
B. Indicate we assume that some ethical hackers may experience some difficulties in retrieving their payout as part of a responsible disclosure protocol.
II. Indicate this presentation merely presents a protocol, not an implementation.
III. Mention the protocol is intended to be open source.
IV. Explain protocol using with visualisation of stakeholder and hacker interaction on the decentralised network.
% TODO: make detailed.
VI. Cover protocol attacks by malicious actors, and accompanying incentives.
% TODO: make detailed.
VII. Mention weaknesses of protocol: 
A. Anyone can see that a zero-day exists (not what it is). 
B. Limited vulnerability scope due to deterministic verifiability requirement. 
C. May incentivise companies to convey a false sense of security by overemphasising the numbers, instead of providing a full picture including misconfigurations, supply-chain attacks, social engineering etc.
VII. Summarise based on which assumptions this protocol can improve market efficiency for deterministically verifiable zero-day exploits. 
IX. Propose strategy to move forward to implementation.

%
%Good Outlines are discussed here:
%https://writingprocess.mit.edu/process/step-2-plan-and-organize/creating-detailed-outline
%https://www.defcon.org/html/links/dc-speakerscorner.html#nikita-cfp
%https://www.defcon.org/html/links/dc-speakerscorner.html#leah-cfp-process
%
%BAD outlines look like these:
%i. intro
%ii. something
%iii. something else
%iv. conclusion
%v. q&a
%
%WHY DEF CON?:
%Note: This is your opportunity to directly address the DEF CON review board as to why you think your talk is good for our hacker con. More and more we are seeing submissions that are well suited for cyber security industry conferences, developer conferences, or local meetups but may not align with the board's view of the DEF CON spirit. The bumper sticker "Keep Infosec out of Hacking" comes to mind. We get many great submissions but they aren't always aimed at the 30,000 hackers at our con. This frequently results in the review board saying "Unfortunately, this is not a DEF CON talk". DEF CON is a hacking con, so how does your content fit in to that culture, spirit, and subject matter?
%
We hope to make the life of hackers a bit simpler. In particular, there is no more discontent about triage, there is no question on legality, attacks can be submitted anonymously, and there are no more employees of companies that need to be convinced of the importance/impact of the vulnerability.
Additionally, we hope that wide-scale adoption of this protocol shows the world that the security of open source software against zero-day exploits, is only as strong as its weakest link, by repeatedly putting a price tag on that weakest link. Such a price tag may make it easier for companies to allocate (more) funds into cybersecurity, we hope some of these funds eventually reach (more) hackers.

%SCHEDULING AND EQUIPMENT REQUIREMENTS
%Is there a specific day or time by which you must present?* Yes or No.
No
%If Yes, Please indicate the dates/times and restriction.
Not applicable.
%Will you require more than 1 projector feed? Yes or No. If yes, please specify how many and why.
No.

%Are there any other special equipment needs that you will require to successfully present your talk?
No.

%SUPPORTING FILES:
%Note: Additional supporting materials such as code, white papers, proof of concept, etc. should be sent along with your email submission. Additional files that may help in the selection process should be included. We are not asking for a complete presentation for this initial submission and full slides will only be required if you are selected for presenting. It is the submitter's responsibility to remove any PII from any attached slides, white papers, or supporting materials, and to appropriately sanitize any metadata in the provided content.
A whitepaper is included in the submission. If time permits, we would greatly appreciate your feedback on our submission and/or whitepaper.
%SUBMISSION AGREEMENTS

%Please read and accept these terms by inserting your name where noted. Failure to do so will render your submission incomplete. Please read these carefully as some of the terms have changed.

%Grant of Copyright Use
%I warrant that the above work has not been previously published elsewhere, or if it has, that I have obtained permission for its publication by DEF CON Communications, Inc. and that I will promptly supply DEF CON Communications, Inc. with wording for crediting the original publication and copyright owner. If I am selected for presentation, I hereby give DEF CON Communications, Inc. permission to duplicate, record and redistribute this presentation, which includes, but is not limited to, the conference proceedings, conference CD, video, audio, and hand-outs to the conference attendees for educational, on-line, and all other purposes.
Yes
%Terms of Speaking Requirements
%1) I will submit a completed presentation, a copy of the tool(s) and/or code(s), and a reference to all of the tool(s), law(s), Web sites and/or publications referenced to at the end of my talk and as described in this CFP submission for publication on the DEF CON media server, to be released the day of the conference, by 12:00 noon Pacific time, July 15, 2022.
Yes
%2) I will submit a final Abstract and Biography for the DEF CON website and Printed Conference Materials by 12:00 noon Pacific time, June 15, 2022.
Yes
%3) I understand if I fail to submit a completed PDF presentation by July 15, 2022, I may be replaced by an alternate presentation or may forfeit my honorarium. This decision will be made by DEF CON and I will be informed in writing of my status.
Yes
%4) I will include a detailed bibliography as either a separate document or included within the presentation of all resources cited and/or used in my presentation.
Yes
%5) I will complete my presentation within the time allocated to me - not running over, or excessively under the time allocation.
Yes
%6) I understand that DEF CON will provide 1 projector feed, 2 screens, microphones, wired and/or wireless Internet. I understand that I am responsible for providing all other necessary equipment, including laptops and machines (with VGA output), to complete my presentation.
Yes
%7) If applicable, I will submit within 5 days of the completion of the conference any updated, revised or additional presentation(s) or materials that were used in my presentation but not included on the conference media server or conference proceedings.
Yes
%Terms of Speaking Remuneration
%1). DEF CON will provide 4 nights hotel per accepted presentation for the primary speaker only. The hotel will be at the DEF CON Venue properties, and of DEF CON’s choosing. I understand I will need to confirm my hotel nights and submit my preferences by the date listed in my official acceptance letter. I understand that I will be responsible for my own travel expenses, unless prior approval is made with special exception.
Yes
%2) I understand that DEF CON will issue one $300 payment per presentation to the primary speaker only. Payment will be made in the form of company check. I may choose to waive my $300 honorarium in exchange for 3 DEF CON Human badges, received at the start of the conference. I may also choose to donate my honorarium to charity.
Yes
%3) I understand that I may receive payment on-site at the conference. If selecting the $300 payment as my honorarium, I must provide a valid name and postal mail address so that the payment may be mailed. In some rare cases, I may be required to complete a W8 (Non-U.S. Citizen) or W9 (U.S. Citizen) before payment is issued.
Yes
%4) I understand that I will be paid within 30 days from the end of the conference, after I have completed my presentation. I understand that should my talk be determined to be unsuitable (e.g. a vendor or sales pitch, a talk on the keeping of goats, etc.) after I have presented, that I will not receive an honorarium.
Yes
%As detailed above, I, Victoria Bosch, have read and agree to the Grant of Copyright Use. I, Victoria Bosch, have read and agree to the Terms of Speaking Requirements. I, Victoria Bosch, have read and agree to the Agreement to Terms of Speaking Remuneration or I will forfeit my honorarium.
Yes
%PRESS CONTACT
I, Victoria Bosch understand that DEF CON's official Press Liaison & Staff may contact me. I consent to be contacted in order to arrange interviews with the media. My contact information will not be given to third parties without my consent.
%< OR >
%No, I (insert primary speaker name), don't want to be contacted by DEF CON's press staff for any reason. Our policies, including our privacy policy are located here: https://www.defcon.org/html/links/dc-policy.html
%
%Top of Page
\end{document}