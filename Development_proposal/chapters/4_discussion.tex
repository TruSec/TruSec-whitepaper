\section{Discussion}
\label{sec:discussion}
The presented proposal for protocol development can be critically evaluated. This section aims to identify possible weak points.
\subsection{Limitations}
The following limitations are identified in the protocol:
\begin{enumerate}
    \item The proposed protocol, in its initial form, does not (necessarily) work for security compromises that are not clearly pre-defined. For example, if the decentralised virtual machine/stack/honeypot is configured to only pay-out in case an internal value/secret is modified, a whitehat hacker might be able to gain read-access to the secret, which could be considered a hack, but the whitehat hacker would not receive a payout. Accordingly, companies may specify different payouts to different types of security breaches. This may reduce the added value of collaborative staking.
    \item The costs of running a decentralised virtual machine, along with their interactions, are currently expected to be relatively high based on e.g. the costs of roughly 50 dollars for a single Ethereum transaction.
    \item We expect most hacks do not rely on pure zero-day exploits, accordingly we think the scope of this protocol is significantly limited w.r.t. the complete cybersecurity threat landscape.
    \item This protocol will most likely not allow companies to test their entire system, as we currently consider it practically infeasible to simulate the various types of social engineering and or interactions with non-decentralised platforms (on a blockchain). So companies cannot make claims about their overall level of cybersecurity based on this protocol alone.
    \item This protocol does not protect against economically irrational malicious agents. Examples could be:
    \begin{enumerate}
        \item Actors with revenge sentiment. They could for example skip the payout and use zero-day exploits to hurt a company that staked their open source software stack.
        \item Nation states may not care about payouts and instead use found zero-day exploits themselves, instead of disclosing them.
    \end{enumerate}
\end{enumerate}

\subsection{Related Work}
It was noted during the TechEx conference, that companies like Google and Microsoft already fund vulnerability disclosures for, for example, Ubuntu. This can be seen as collective funding, hence one could argue the added value of the proposed protocol may be limited in this respect.

Additionally, there are companies like HackerOne that perform independent triage, hence one could argue the added value of doing this in a decentralised fashion is limited.