\documentclass[a4paper, UKenglish, cleveref, autoref, thm-restate]{lipics-v2021}
%This is a template for producing LIPIcs articles. 
%See lipics-v2021-authors-guidelines.pdf for further information.
%for A4 paper format use option "a4paper", for US-letter use option "letterpaper"
%for british hyphenation rules use option "UKenglish", for american hyphenation rules use option "USenglish"
%for section-numbered lemmas etc., use "numberwithinsect"
%for enabling cleveref support, use "cleveref"
%for enabling autoref support, use "autoref"
%for anonymousing the authors (e.g. for double-blind review), add "anonymous"
%for enabling thm-restate support, use "thm-restate"
%for enabling a two-column layout for the author/affilation part (only applicable for > 6 authors), use "authorcolumns"
%for producing a PDF according the PDF/A standard, add "pdfa"

%\pdfoutput=1 %uncomment to ensure pdflatex processing (mandatatory e.g. to submit to arXiv)
%\hideLIPIcs  %uncomment to remove references to LIPIcs series (logo, DOI, ...), e.g. when preparing a pre-final version to be uploaded to arXiv or another public repository

%\graphicspath{{./graphics/}}%helpful if your graphic files are in another directory

\bibliographystyle{plainurl}% the mandatory bibstyle
\usepackage{float}


\usepackage{amsmath} % need to be on top for eps files
\usepackage{mathtools}

\usepackage{graphicx}

\usepackage{cleveref}

\usepackage{algorithmicx}

\usepackage{algcompatible}
\crefname{algocf}{alg.}{algs.}
\Crefname{algocf}{Algorithm}{Algorithms}
\usepackage[ruled,vlined,linesnumbered]{algorithm2e}
\usepackage[ruled,vlined,linesnumbered]{algorithm2e}
\renewcommand{\algorithmiccomment}[1]{\bgroup\hfill\tiny//~#1\egroup}
\newcommand\bigO[1]{$\mathcal{O}(#1)$}

\crefname{lstlisting}{listing}{listings}
\Crefname{lstlisting}{Listing}{Listings}

\title{Development Proposal: Security Variant of TruCol protocol} %TODO Please add
\subtitle{Towards a 2022 Blackhat (USA) presentation}
%\titlerunning{Spiking Minimum Dominating Set Approximation} %TODO optional, please use if title is longer than one line

%\titlerunning{Dummy short title} %TODO optional, please use if title is longer than one line
% TODO: remove names from print for double blind article
\author{Chihab Amghane}{Radboud University}%, [optional: Address], Country \and My second affiliation, Country \and \url{http://www.myhomepage.edu} }
{}{}{(Optional) author-specific funding acknowledgements}%TODO mandatory, please use full name; only 1 author per \author macro; first two parameters are mandatory, other parameters can be empty. Please provide at least the name of the affiliation and the country. The full address is optional

\author{Victoria Bosch}{Radboud University}%, [optional: Address], Country \and My second affiliation, Country \and \url{http://www.myhomepage.edu} }
{}{}{}%TODO mandatory, please use full name; only 1 author per \author macro; first two parameters are mandatory, other parameters can be empty. Please provide at least the name of the affiliation and the country. The full address is optional

\author{Rashim Charles}{Radboud University}%, [optional: Address], Country \and My second affiliation, Country \and \url{http://www.myhomepage.edu} }
{}{}{}%TODO mandatory, please use full name; only 1 author per \author macro; first two parameters are mandatory, other parameters can be empty. Please provide at least the name of the affiliation and the country. The full address is optional

\author{Marc Droogh}{Delft University of Technology}%, [optional: Address], Country \and My second affiliation, Country \and \url{http://www.myhomepage.edu} }
{}{}{}%TODO mandatory, please use full name; only 1 author per \author macro; first two parameters are mandatory, other parameters can be empty. Please provide at least the name of the affiliation and the country. The full address is optional

\author{Clara Main}{Radboud University}%, [optional: Address], Country \and My second affiliation, Country \and \url{http://www.myhomepage.edu} }
{}{}{}%TODO mandatory, please use full name; only 1 author per \author macro; first two parameters are mandatory, other parameters can be empty. Please provide at least the name of the affiliation and the country. The full address is optional

\author{Chinari Subhechha Subudhi}{Delft University of Technology}%, [optional: Address], Country \and My second affiliation, Country \and \url{http://www.myhomepage.edu} }
{}{}{}%TODO mandatory, please use full name; only 1 author per \author macro; first two parameters are mandatory, other parameters can be empty. Please provide at least the name of the affiliation and the country. The full address is optional

\author{Akke Toeter}{Delft University of Technology, Radboud University}%, [optional: Address], Country \and My second affiliation, Country \and \url{http://www.myhomepage.edu} }
{a.h.h.toeter@student.tudelft.nl}{0000-0002-9577-920X}{}%TODO mandatory, please use full name; only 1 author per \author macro; first two parameters are mandatory, other parameters can be empty. Please provide at least the name of the affiliation and the country. The full address is optional

\author{Eric van der Toorn}{Delft University of Technology}%, [optional: Address], Country \and My second affiliation, Country \and \url{http://www.myhomepage.edu} }
{}{}{}%TODO mandatory, please use full name; only 1 author per \author macro; first two parameters are mandatory, other parameters can be empty. Please provide at least the name of the affiliation and the country. The full address is optional

% \author{V Bosch}{Radboud}{email}{[orcid]}

% \author{A Diehl}{Radboud}{email}{[orcid]}

% \author{A Toeter}{Radboud}{email}{[orcid]}

% \author{Kwisthout}{Radboud}{email}{[orcid]}


\authorrunning{ } %TODO mandatory. First: Use abbreviated first/middle names. Second (only in severe cases): Use first author plus 'et al.'

\Copyright{ } %TODO mandatory, please use full first names. LIPIcs license is "CC-BY";  http://creativecommons.org/licenses/by/3.0/

\ccsdesc[500]{Computing methodologies~Distributed algorithms}
\ccsdesc[500]{Computer systems organization~Neural networks}

% \ccsdesc[100]{\textcolor{red}{Replace ccsdesc macro with valid one}} %TODO mandatory: Please choose ACM 2012 classifications from https://dl.acm.org/ccs/ccs_flat.cfm 

\keywords{Neuromorphic Computing, Spiking Neural Network, Graph Algorithms, Distributed Computing, Minimum Dominating Set} %TODO mandatory; please add comma-separated list of keywords

\category{} %optional, e.g. invited paper

\relatedversion{} %optional, e.g. full version hosted on arXiv, HAL, or other respository/website
%\relatedversiondetails[linktext={opt. text shown instead of the URL}, cite=DBLP:books/mk/GrayR93]{Classification (e.g. Full Version, Extended Version, Previous Version}{URL to related version} %linktext and cite are optional

%\supplement{}%optional, e.g. related research data, source code, ... hosted on a repository like zenodo, figshare, GitHub, ...
%\supplementdetails[linktext={opt. text shown instead of the URL}, cite=DBLP:books/mk/GrayR93, subcategory={Description, Subcategory}, swhid={Software Heritage Identifier}]{General Classification (e.g. Software, Dataset, Model, ...)}{URL to related version} %linktext, cite, and subcategory are optional

%\funding{(Optional) general funding statement \dots}%optional, to capture a funding statement, which applies to all authors. Please enter author specific funding statements as fifth argument of the \author macro.

\acknowledgements{I want to thank \dots}%optional

% \nolinenumbers %uncomment to disable line numbering

\graphicspath{{latex/publication_opodis_2021/images/}}


%Editor-only macros:: begin (do not touch as author)%%%%%%%%%%%%%%%%%%%%%%%%%%%%%%%%%%
\EventEditors{Akke Toeter}
\EventNoEds{2}
\EventLongTitle{}
\EventShortTitle{TruSec}
\EventAcronym{}
\EventYear{2021}
\EventDate{December 23, 2021}
\EventLocation{Delft, Nijmegen}
\EventLogo{}
\SeriesVolume{2}
\ArticleNo{2}
%%%%%%%%%%%%%%%%%%%%%%%%%%%%%%%%%%%%%%%%%%%%%%%%%%%%%%

\begin{document}

\maketitle
\section{Introduction}
\label{sec:introduction}
This document presents a planning proposal for the development of a Trustless Security protocol with the short-term purpose of presenting the protocol during the 2022 Blackhat (USA) conference. The aim of the proposed security proposal is to help ethical hackers retrieve their bounties without ambiguity, whilst simultaneously enabling companies to show their customers how much money is staked on their open source software stacks being secure.

To explain how the protocol may help both of these stakeholders (ethical hackers and companies using open source software), we will first describe, what we think is, the status quo. Then we will explain how the protocol changes, what we think is, the status quo. This will be done in \cref{sec:protocol}. Next, \cref{sec:implementation} describes strategies to specify how the protocol may be implemented. The limitations and weaknesses of our strategy and protocol are detailed in \cref{sec:discussion}. Since we are relatively new to the field of cyber security, we would like to ask feedback on:
\begin{itemize}
	\item The validity of our assumptions.
	\item The added value of this protocol in real life applications.
	\item Any perspectives we might have overlooked.
\end{itemize}
These questions are specified in \cref{sec:questions}. The information used to generate a planning towards the Blackhat presentation is included in \cref{sec:planning}. A brief conclusion to this proposal is presented in \cref{sec:conclusion}.
\section{Protocol}
\label{sec:protocol}

\section{Implementation}\label{sec:implementation}

\section{Discussion}
\label{sec:discussion}
The presented proposal for protocol development can be critically evaluated. This section aims to identify possible weak points.
\subsection{Limitations}
The following limitations are identified in the protocol:
\begin{enumerate}
    \item The proposed protocol, in its initial form, does not (necessarily) work for security compromises that are not clearly pre-defined. For example, if the decentralised virtual machine/stack/honeypot is configured to only pay-out in case an internal value/secret is modified, a whitehat hacker might be able to gain read-access to the secret, which could be considered a hack, but the whitehat hacker would not receive a payout. Accordingly, companies may specify different payouts to different types of security breaches. This may reduce the added value of collaborative staking.
    \item The running a decentralised virtual machine with a high degree of decentralisation, along with their interactions is expected to be costly. This expectation is based on approximate costs of roughly 50 dollars for a single Ethereum transaction \cite{todo}.
    \item We expect most hacks do not rely on pure zero-day exploits, accordingly we think the scope of this protocol is significantly limited w.r.t. the complete cybersecurity threat landscape.
    \item This protocol will most likely not allow companies to test their entire system, as we currently consider it practically infeasible to simulate the various types of social engineering and or interactions with non-decentralised platforms (on a blockchain). So companies cannot, in good conscience, make claims about their overall level of cybersecurity based on this protocol alone.
    \item This protocol does not protect against economically irrational malicious agents. Examples could be:
    \begin{enumerate}
        \item Actors with revenge sentiment. They could for example skip the payout and use zero-day exploits to hurt a company that staked their open source software stack.
        \item Nation states may not care about payouts and instead use found zero-day exploits themselves, instead of disclosing them.
    \end{enumerate}
\end{enumerate}

\subsection{Related Work}
It was noted during the TechEx conference, that companies like Google and Microsoft already fund vulnerability disclosures for, for example, Ubuntu. This can be seen as collective funding, hence one could argue the added value of the proposed protocol may be limited in this respect.

Additionally, there are companies like HackerOne that perform independent triage, hence one could argue the added value of doing this in a decentralised fashion is limited.%\newpage
\section{Questions}
\label{sec:questions}
%The advice and expert knowledge within F-secure is asked in particular on the following questions:
Your advice and expert knowledge within the domain of cybersecurity and ethical hacking is asked in particular on the following questions:
\begin{enumerate}
    \item Would you perhaps be able to give us an approximation on the $V,W,X,Y,Z$ percentages of cyberattacks as displayed in \cref{fig:protocol_scope}? 
    \begin{enumerate}
        \item Note, we will have to do our own due-diligence on this, however, as a first indicator/ballpark estimate, your perspective would significantly move us forward in assessing the (potential) real-world impact of the proposed protocol. 
        \item Are there relevant attack strategies that we omitted? (That are perhaps (indirectly) suitable for the proposed protocol).
    \end{enumerate}
    \item Based on your experience, would you expect the protocol to be of value in real-world applications?
    \begin{enumerate}
        \item (If not), which bottlenecks do you identify?
    \end{enumerate}
    \item Would you consider a talk at Blackhat feasible (assuming the work is done well), with as topic: a presentation of the protocol (specification) with/without working implementation?
    \begin{enumerate}
        \item Note, understand you do not have an crystal ball, however, perhaps your team has more experience into typical Blackhat submission topics and trends, than us.
    \end{enumerate}
    \item Are there perspectives that you would like to share? Did we miss any angles/relevant factors? Is there any advice you could give us, or research-directions that may be relevant in this endeavour?
\end{enumerate}%\newpage
\section{Planning}
\label{sec:planning}
This section summarises the information required to generate a planning towards a Call for Papers submission for the \textit{Policy Track} of the Blackhat 2022 USA submission.


\subsection{Schedule}
The 2022 Blackhat (USA) edition takes place on 2022-08-06 to 2022-08-11. Even though the call for papers is not yet open, one can develop a planning analog to the call for papers for the 2021 Blackhat USA edition. For that conference, the dates were specified as:\\
Source: \url{https://www.blackhat.com/us-21/call-for-papers.html}
\begin{itemize}
	\item Call for Papers Opened: February 2, 2021
	\item Call for Papers Closed: April 10, 2021
	\item Notification to Submitters: end of May, 2021
	\item Event Dates: July 31 - August 5, 2021
\end{itemize}
Hence, shifting the planning with one week, since the 2022 edition will occur one week later:
\begin{itemize}
	\item Call for Papers Opened: February 9, 2021
	\item Call for Papers Closed: April 17, 2021
	\item Notification to Submitters: end of May, 2021
	\item Event Dates: August 06 - August 11, 2022
\end{itemize}

\subsection{Deliverables}
To create a succesfull submission to the Blackhat 2022 (USA) edition, the following deliverables are required:
\begin{enumerate}
	\item A track specification.
	\begin{itemize}
		\item Source: \url{https://www.blackhat.com/html/tracks.html}
	\end{itemize}
	
	\item \textit{Assumed:} Abstract specification
	\begin{itemize}
		\item Source: \url{https://i.blackhat.com/docs/cfp-sample-submissions.pdf}
	\end{itemize}
	
	\item \textit{Assumed:} Presentation Outline
	\begin{itemize}
		\item Source: \url{https://i.blackhat.com/docs/cfp-sample-submissions.pdf}
	\end{itemize}

	\item \textit{Assumed:} Attendee Takeaways
	\begin{itemize}
		\item Source: \url{https://i.blackhat.com/docs/cfp-sample-submissions.pdf}
	\end{itemize}

	\item \textit{Assumed:} Why Black Hat motivation.
	\begin{itemize}
		\item Source: \url{https://i.blackhat.com/docs/cfp-sample-submissions.pdf}
	\end{itemize}
	
	\item \textit{Assumed:} Presentation slides.
	\begin{itemize}
		\item Source: Imagination.
	\end{itemize}
\end{enumerate}


\subsubsection{Submission Requirements (ASIA)}
The following Blackhat submission requirements are specified for the ASIA event:\\
Source: \url{https://www.blackhat.com/call-for-papers.html}
\begin{enumerate}
	\item Submissions may only be entered by researchers/speakers (no submissions from PR firms/marketing representatives).
	\item Black Hat does not accept product or vendor-related pitches. Black Hat will disqualify any product or vendor pitch.
	\item Submissions must clearly detail the concepts, ideas, findings, and solutions a researcher or speaking team plans to present.
	\item Submissions that highlight new research, tools, vulnerabilities, etc. will be given priority.
	\item Submissions that include White Papers are highly encouraged and will also be given priority.
	\item Black Hat will disqualify incomplete submissions; complete your submission in its entirety.
	\item Individuals may submit more than one proposal, but each proposal must be submitted via a separate submission form.
	\item Each submission must include detailed biographies of the proposed speaking team.
	\item Submitters will be contacted directly if Review Board members have any questions about a submission.
\end{enumerate}

\subsubsection{Tailoring Submission}
Suggested resources to tailor the submission to maximise acceptance probability:\\
Source: the recommendation section of: \url{https://asia-briefings-cfp.blackhat.com/}.
\begin{itemize}
	\item Example submissions: \url{https://i.blackhat.com/docs/cfp-sample-submissions.pdf}
	\item Tips: \url{https://insinuator.net/2017/04/some-quick-tips-for-submitting-a-talk-to-black-hat-or-troopers/}
	\item Acceptance: \url{https://www.helpnetsecurity.com/2016/03/30/how-to-get-your-talk-accepted-at-black-hat/}
	\item Tips: \url{https://hexsec.blogspot.com/2012/12/create-good-security-cfp-responses.html}
	\item Pitfall avoidance: \url{https://research.kudelskisecurity.com/2020/04/02/5-common-cfp-submission-mistakes-for-security-conferences/}
\end{itemize}%\newpage
\section{Conclusion and Recommendations}
\label{sec:conclusion}
The proposed protocol may enable companies to convey the level of security of (segments of) their open source technology stack more intuitively to their customers/stakeholders. Additionally, the protocol enables whitehat/ethical hackers to retrieve payouts directly without ambiguity. 

The development of the protocol requires significant work, and it is currently not clear what the added value of the protocol would be in real-life settings. A presentation of the protocol, without working implementation, in the \textit{Policy Track} of the Blackhat (USA) 2022 edition may be a direct probe to the cybersecurity world to assess the interest in actually implementing the protocol.
%% Summarize the other chapters
%% Explain what TruSec does

%% Write how TruSec works
%% Summarise implementation
%% Summarize security issues
%% Summarise discussion
%% Mention the protocol can promote a fair and inclusive TDD market contributing to SDG 8.



%\subsection{Recommendations}
%% Include the possible impact of the protocol on society
%The following recommendations are made:
%\begin{enumerate}[label=\arabic*]
    %% To mitigate edge cases
%    \item \label{rec:ecosystem} 
%    \item \label{rec:prelauch_period}
%    \item \label{rec:security_tools}
%\end{enumerate}
%\bibliography{references.bib}
 

 \end{document}