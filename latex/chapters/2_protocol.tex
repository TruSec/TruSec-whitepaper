\section{Protocol}
\label{sec:protocol}
This section presents the TruSec protocol, and explains how it can improve the way vulnerability disclosures are completed for deterministically verifiable zero-day exploits.

\subsection{Scope}\label{subsec:scope}
The protocol is primarily designed to automate deterministically verifiable zero-day exploits vulnerability disclosures. It can also be used to hedge against misconfigurations and supply-chain attacks. For example, companies can add a specific configuration (yaml) to the DVM, and add a bounty on that forked DVM. This way, a hacker may leverage the particular configuration to find an exploit. This procedure also allows the protocol to identify some supply-chain attack vulnerabilities. For example, if an invalid certificate is used to compromise the device.

However, both misconfiguration and supply chain attack partially deviate from the main benefit of collective nature of the protocol. For example, it may incentivise hackers to focus efforts on particular configurations, that are not (necessarily) useful for other companies. However, at the same time, hackers could still opt to focus on the mutual elements of all forked decentralised virtual machines to collect the bounties with a single, more powerful exploit. This scope/applicability of the protocol is visualised in \cref{fig:protocol_scope}.
\begin{figure}[H]
    \centering
    \ifx\homepath\overleafhome
        \includegraphics[width=0.50\textwidth]{Images/Diagrams/scope.png}
    \else
        \includegraphics[width=0.50\textwidth]{latex/Images/Diagrams/scope.png}
    \fi
    \caption{The proposed TruSec protocol is not suited to deal with social engineering attacks, nor is it ideal for misconfiguration exploits and/or supply-chain attacks. Instead, it is designed to increase the rate of discovery of deterministically verifiable zero-day exploits. Note, we acknowledge that attacks can be, and often are, a combination of the types.}
    \label{fig:protocol_scope}
\end{figure}
With respect to \cref{fig:protocol_scope}, the following notes are made:
\begin{enumerate} 
    \item The orange attack types imply the proposed protocol is not designed to tackle these issues, nor does it provide full coverage (against malicious agents) for these attack types. However:
    \begin{enumerate}
        \item The misconfiguration could be covered if companies upload their configurations into DVMs. These configurations would typically not benefit from the collaborative staking, as it is less likely that other companies happen to use the same configurations.
        \item Some of the supply chain attacks could be covered if the ethical hackers are able to propagate these supply chain exploits into the DVMs.
    \end{enumerate}
\end{enumerate}

\subsection{Usage}
\noindent With this scope defined, one can look at how companies and ethical hackers interact according to the proposed protocol.

\noindent The basic idea is that companies and users (stakeholders) can put their open source software stacks on a decentralised virtual machine (DVM). They can then collectively stake money on the security of the stacks, such that everyone can see how much money says: \textit{the use of certain software packages/combinations is safe}. This enables companies, to show their customers for example:

\noindent \textit{With us, your data is stored using MongoDB Version 5.1, \$314.159,- says it is uncompromised, and it's running on Ubuntu Server version 21.10, which has \$4.200.000,- staked on its security. This setup has a configuration with a security on which we staked \$9001,-. If any of these software packages get compromised by ethical hackers, \underline{we will be the first to know.}}

\noindent We believe that might be clear language that enables decision makers and customers interested in company $A$, to get a simple, and intuitive understanding on: \textit{how secure} is the software stack of company A?

\noindent For the ethical/ethical hackers, the advantages are clear; they know before they start their work how large their payout will be, and they get a direct payout upon completion of their work (after the predetermined responsible disclosure period has ended).

\subsubsection{Disclaimer}
The presented protocol does not provide insight in the complete security of a system/company. As visualised in \cref{fig:protocol_scope}, the protocol does not cover all attack surfaces of companies. Hence, if other attack surfaces, such as social engineering are used, companies can still get compromised, regardless of the amount they staked. Therefore, it is important that the numerical value of the amount staked on the zero-day exploit security level is not abused to convey a false sense of security by the staking companies to their customers.

\subsection{Description}
The protocol is shown in \cref{fig:interaction}.
\begin{figure}[H]
    \centering
    \ifx\homepath\overleafhome
        \includegraphics[width=1.0\textwidth]{Images/Diagrams/interaction.png}
    \else
        \includegraphics[width=1.0\textwidth]{latex/Images/Diagrams/interaction.png}
    \fi
    
    \caption{Visualisation of the interaction of the TruSec protocol. This is an ever-lasting cycle, where at the end of the process, companies can re-deploy the patched decentralised stack, and allocate new funds. ethical hackers can scan for new attacks.}
    \label{fig:interaction}
\end{figure}
\noindent 

\noindent To summarize, the protocol enables companies and users of open source software to collectively stake money on the software stacks of their choice. Hackers can see these bounties and write a new zero-day exploit for these staked systems. Next, they can deploy their attack to a decentralised locker that only opens to the stakeholders. Only the hacker and the stakeholders can then see the zero-day exploit within the responsible disclosure time (RDT) they specified. Stakeholders can then patch their systems. After the RDT, the exploit is evaluated, and if it compromises the software stack, the hacker automatically receives the staked bounty. The cycle can then start over, with companies re-deploying their patched systems and applying new stakes to their respective security.


\subsubsection{\Cref{fig:interaction} notes}
With respect to \cref{fig:interaction}, the following notes are made:
\begin{enumerate} 
    \item The attack written by the ethical hacker should be accessible on chain, such that everyone can verify that the attack indeed compromises the decentralised VM/honeypot. This is critical for the automatic payout.
    \item The decentralised locker is used to prevent malicious hackers to inspect/copy the attack before the responsible disclosure period is over.
    \item It would be better if the DVM smart contract specifies the locker location, while granting hackers write-access, and read-access only to (certified) stakeholders. This would prevent other hackers from knowing there is a vulnerability discovered in a certain software stack before the responsible disclosure period is over. We expect such a signal might attract unwanted attention. However, at the time of writing, no mechanism is designed that would prevent people from determining whether a hacker has deployed an attack, even if it is encrypted, whilst still allowing stakeholders to access/read the attack within the responsible disclosure. This weakness is discussed in \cref{sec:discussion}.
\end{enumerate}

\subsection{Incentives}
To convey a better understand of the protocol, some of the incentives are evaluated along with the relevant steps shown in \cref{fig:interaction}. Starting with the RDT, companies and stakeholders need to have sufficient time to patch their systems, that is why they have the liberty to specify the disclosure time they need. Since the attack needs to be private until the maximum specified RDT has passed, a check is performed to verify it is indeed hidden to the public in step 12 of \cref{fig:interaction}. 

\subsubsection{Malicious staking}
Malicious actors could stake a minimum amount to set the maximum RDT to an unreasonable amount, such as 3 centuries. To mitigate this tampering, hackers can select which bounties they want to collect. This allows them to forfeit the trivial stake with unreasonable RDT.

Another act of malicious staking could be to have a malicious agent stake to get access to the attack to abuse it within the RDT. In the purest decentralised form of the protocol there is no defence against this.  To alleviate this concern, different strategies can be pursued. For example, limited access to the zero-day exploits can be granted to a ring of trusted companies using self-sovereign identity. This would however introduce subjectivity into the protocol, and it would lower the decentralised nature of the protocol, even when it is done in the form of a DAO. Alternatively, access to the exploit within RDT could be granted to the highest staker only, whilst requiring stakers that want early access to identify themselves using SSI. This would still allow malicious actors to "buy" the exploits within the RDT, like they can currently. However, it would at least inform relevant companies of their exposed position and allows them to outbid the malicious actors.
\subsubsection{Malicious Exploit Publications}
Hackers could also try to sabotage the protocol. For example, they could submit invalid solutions to clog the network. To prevent this, step 8 requires the hacker to deposit evaluation costs in order to be eligible for a payout.

Another attack from a hacker perspective could be to deny access to the stakers. This is why the companies should present a public key in steps 1 and 2, why the hacker should provide access for these keys in step 6, and the DVM smart contract should verify this access in steps 13 and 14.

Another form of malicious exploit publication would be to claim the bounty using the protocol and then to publish the exploit somewhere else anonymously within the RDT. The purely decentralised form has no defence against this behaviour. If it occurs frequently, a DAO/voting mechanism could be implemented that determines whether the exploit has remained hidden to the public, however this introduces subjectivity into the protocol and opens new kinds of attack surfaces. 

\subsection{Added Value}
The protocol allows companies to show their users how secure their open sources software stacks are against deterministically verifiable zero-day exploits. This can be done by showing the users how much money is "staked" on the security of their respective systems. This simplifies the comparison that customers can make between the security of companies against deterministically verifiable zero-day exploits. Additionally, companies (as well as the users of these open source software stacks) can re-adjust their funds and resource allocations regarding cyber-security based on this insight. This mechanism could, over multiple cycles, result in a more predictable zero-day exploit landscape.

To improve the ease of use and practical application of the protocol, a variant could be written that allows for staking on decentralised containers instead of decentralised virtual machines. This could reduce the required computational resources and broaden the usecases of the protocol to containerized applications (with minimal attack surfaces). Additionaly, using hardware emulators such as QEMU, in decentralised format, companies could use this protocol to hedge against certain types of hardware exploits as well. We hope this allows extending the TruSec protocol to show consumers \textit{"how secure"} their phones are, in real-time.