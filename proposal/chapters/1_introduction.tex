\section{Introduction}
\label{sec:introduction}
This document presents a planning proposal for the development of a Trustless Security protocol with the short-term purpose of presenting the protocol during the 2022 Blackhat (USA) conference. The aim of the proposed security proposal is to help ethical hackers retrieve their bounties without ambiguity, whilst simultaneously enabling companies to show their customers how much money is staked on their open source software stacks being secure.

To explain how the protocol may help both of these stakeholders (ethical hackers and companies using open source software), we will first describe, what we think is, the status quo. Then we will explain how the protocol changes, what we think is, the status quo. This will be done in \cref{sec:protocol}. Next, \cref{sec:implementation} describes strategies to specify how the protocol may be implemented. The limitations and weaknesses of our strategy and protocol are detailed in \cref{sec:discussion}. Since we are relatively new to the field of cyber security, we would like to ask feedback on:
\begin{itemize}
	\item The validity of our assumptions.
	\item The added value of this protocol in real life applications.
	\item Any perspectives we might have overlooked.
\end{itemize}
These questions are specified in \cref{sec:questions}. The information used to generate a planning towards the Blackhat presentation is included in \cref{sec:planning}. A brief conclusion to this proposal is presented in \cref{sec:conclusion}.